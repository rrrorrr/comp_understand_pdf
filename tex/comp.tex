\documentclass{jarticle}

\begin{document}

\title{引き継ぎ資料のようななにか}
\author{鹿児島大学ロボット研究会}

\maketitle

\tableofcontents
\clearpage

\section{はじめに}
本資料は我々趣味の全てについてまとめたものです。
誤字脱字誤解浅解については実験レポートよりも厳しく確認するつもりですが
もしなにかあれば代表者のtwitterアカウント@bot973888までお願いします。
\clearpage
\section{制御}
   \subsection{HelloWorld}
   \subsection{Lチカ}
   \subsection{入力}
   \subsection{モーター}
   \subsection{センサー}
   \subsection{割り込み}
   \subsection{エンコーダ}
   \subsection{Linux}
   \subsection{ROS}
      1.ROSwikiを見ます
      2.終わり
\clearpage
\section{回路}
   \subsection{モータードライバ}
      \subsubsection{リレー型}
      \subsubsection{リレートランジスタ型}
      \subsubsection{NP混合型}
      \subsubsection{フルN型}
\clearpage
\section{競プロ}
   \subsection{アルゴリズム}
      \subsubsection{二分探索}
      \subsubsection{三分探索}
   \subsection{データ構造}
   \subsection{数学}
      \subsubsection{半環問題}
\clearpage
\section{OS}
\clearpage
\section{圏論}
\clearpage
\section{おすすめアニメ}
   \subsection{serial experiments lain}
      スーパーハカーのほのぼの日常系アニメ
   \subsection{なるたる}
      不思議な生物たちが存在する世界の中で繰り広げられる日常を描いたハートフルアニメ。漫画も見よう!
   \subsection{NHKへようこそ}
      ニートを更生させるため少女が奮闘する日常系アニメ
   \subsection{ぼくらの}
      わるい怪獣からぼくらの町を守りぬけ!ロボット系アニメ
   \subsection{School Days}
      ドキドキの学園生活を楽しめる学園系アニメ
   \subsection{メイドインアビス}
      ナナチ
\end{document}
