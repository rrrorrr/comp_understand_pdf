\documentclass{jarticle}
\usepackage{listings}
\begin{document}

\title{引き継ぎ資料のようななにか}
\author{鹿児島大学ロボット研究会}

\maketitle

\tableofcontents
\clearpage

\section{はじめに}
本資料は我々いもほり研究会メンバーの趣味の全てについてまとめたものです。
誤字脱字誤解浅解については実験レポートよりも厳しく確認するつもりですが
もしなにかあれば代表者のtwitterアカウント@bot973888までお願いします。

著者一覧:@bot973888、@guestman000
\clearpage

\section{制御}
   プログラミング言語はC,C++を扱います。
   \subsection{HelloWorld}
      1.統合開発環境を入れます。
      2.関数ってなんだ?.pptxを見ます。
      3.闇の世界を開いて終わり(閉じるとペナルティ)
      
   \begin{lstlisting}
      int main(){
         return 0;
      }
   \end{lstlisting}
   以上のコードは何も行わないコードです。
   プログラムはint main()の後ろに在る中括弧の中身を実行します。
   これはどんなプログラムに対してもそうなので
   他人のプログラムを見るときは
   まず最初に、このint main()の中身を見ます。
   それと、いつまでもint main()と呼ぶのは不便ですから
   今からこれをmain関数と呼ぶことにします。
   
     
   ・プログラムはmain関数の中括弧の中を実行する。
  
  
   次に気になるのは return 0; です。
   return 0; は main関数の終了を表していると考えてください。
   また後ろに付いているセミコロンは毎回命令の後ろにつけなければいけません。
   今回は return 0 という命令の後ろにくっついていることになります。


   ・return 0; は main関数の終了を表す。

   \subsection{Lチカ}
      マイコンでの電圧と時間の概念とアナログ出力とデジタル出力の違いを学びましょう。
   \subsection{入力}
   \subsection{モーター}
      アナログ出力を忘れずに回路班に土下座しましょう。
   \subsection{センサー}
      抵抗を忘れずに挟んで仕様書先生の言うことをよく聞いて電流を計測値に直しましょう。
   \subsection{割り込み}
   \subsection{エンコーダ}
   \subsection{フィードフォワード制御}
   \subsection{フィードバック制御}
      \subsubsection{P}
      \subsubsection{PI}
      \subsubsection{PD}
      \subsubsection{PID}
   \subsection{Linux}
   \subsection{ROS}
      1.ROSwikiを見ます
      2.rosserial_Arduinoでキャッキャツウシンソクドマチガットルヤンケー!します
      3.終わり
      
\clearpage

\section{回路}
   \subsection{モータードライバ}
      \subsubsection{リレー型}
      \subsubsection{リレートランジスタ型}
      \subsubsection{NP混合型}
      \subsubsection{フルN型}
\clearpage

\section{競プロ}
   \subsection{オーダー表記}
      プログラムの早さをアバウトに表す方法です。
      \\次のコードを例にとってオーダー表記をしてみましょう。
      \begin{lstlisting}[basicstyle=\ttfamily,frame=single]
         int main(){
            int N;
            scanf("%d",&N);
            for(int i=0;i<N;i++){
               printf("HelloWorld");
            }
            return 0;
         }
      \end{lstlisting}
   \subsection{アルゴリズム}
      \subsubsection{全探索}
      \subsubsection{二分探索}
      \subsubsection{三分探索}
   \subsection{データ構造}
   \subsection{数学}
      \subsubsection{半環問題}
\clearpage

\section{OS}
\clearpage

\section{圏論}
\clearpage

\section{広報}
   \subsection{blog}
      URL http://kadairobo.blog111.fc2.com/
      タグにできるだけ自分の名前(半値でもok)つけてね
   \subsection{homepage}
\clearpage

\section{おすすめアニメなど}
   \subsection{serial experiments lain}
      スーパーハカーのほのぼの日常系アニメ
   \subsection{なるたる}
      不思議な生物たちが存在する世界の中で繰り広げられる日常を描いたハートフルアニメ。漫画も見よう!
   \subsection{NHKへようこそ}
      ニートを更生させるため少女が奮闘する日常系アニメ
   \subsection{ぼくらの}
      わるい怪獣からぼくらの町を守りぬけ!ロボット系アニメ
   \subsection{School Days}
      ドキドキの学園生活を楽しめる学園系アニメ
   \subsection{メイドインアビス}
      ナナチ
   \subsection{ステラのまほう}
      集団製作と創作の苦しみ、楽しさ、つらさ、人間関係の苦しみ、つらさ、複雑さを学べるキラキラアニメです(ただし徹夜は息を吸うようにする)
      漫画では昼ドラ真っ青の人間関係
   \subsection{SSSS.GRIDMAN}
      戦闘シーン中毒になっちゃいそうよ!
   \subsection{王様ゲーム The Animations}
      ダイナミックコードに埋もれた悲劇の迷作。是非行き詰ったときはこれを見て虚無に陥ってほしい
      特に意味深な登場したやつが登場からたった2話で燃えながらPC弄ってワクチンプログラムを作る姿は皆見習ってほしい
\end{document}
